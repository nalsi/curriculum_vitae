%%%%%%%%%%%%%%%%%%%%%%%%%%%%%%%%%%%%%%%%%
% Medium Length Graduate Curriculum Vitae
% LaTeX Template
% Version 1.1 (9/12/12)
%
% This template has been downloaded from:
% http://www.LaTeXTemplates.com
%
% Original author:
% Rensselaer Polytechnic Institute (http://www.rpi.edu/dept/arc/training/latex/resumes/)
%
% Important note:
% This template requires the res.cls file to be in the same directory as the
% .tex file. The res.cls file provides the resume style used for structuring the
% document.
%
%%%%%%%%%%%%%%%%%%%%%%%%%%%%%%%%%%%%%%%%%

%----------------------------------------------------------------------------------------
%	PACKAGES AND OTHER DOCUMENT CONFIGURATIONS
%----------------------------------------------------------------------------------------

\documentclass[margin, 10pt]{res} % Use the res.cls style, the font size can be changed to 11pt or 12pt here

\usepackage[breaklinks=true]{hyperref}

\setlength{\textwidth}{5in} % Text width of the document
\pagestyle{plain}

\begin{document}

%----------------------------------------------------------------------------------------
%	NAME AND ADDRESS SECTION
%----------------------------------------------------------------------------------------

\moveleft.5\hoffset\centerline{\large\bf Kai Li, PhD} % Your name at the top
 
\moveleft\hoffset\vbox{\hrule width\resumewidth height 1pt}\smallskip % Horizontal line after name; adjust line thickness by changing the '1pt'
 
\moveleft.5\hoffset\centerline{\href{mailto:kli16@utk.edu}{kli16@utk.edu}}
\moveleft.5\hoffset\centerline{ORCID: \href{https://doi.org/10.1177/0165551522110183}{\nolinkurl{0000-0002-7264-365X}} }

%----------------------------------------------------------------------------------------

\begin{resume}


\section{RESEARCH INTERESTS}  

Scholarly communication; Quantitative science studies; Digital humanities; Information organization; Science and technology studies; Information visualization

\section{ACADEMIC POSITION}

{\sl Assistant Professor} \hfill 2023 -  \\
School of Information Sciences\\
\textbf{University of Tennessee, Knoxville}, Tennessee, USA

{\sl Assistant Professor} \hfill 2020 - 2022 \\
School of Information Resource Management\\
\textbf{Renmin University of China}, Beijing, China

\section{EDUCATION}

{\sl Ph.D. in Information Studies} \hfill 2015 - 2020 \\
College of Computing \& Informatics\\
\textbf{Drexel University}, Philadelphia, PA
\begin{itemize}
\item Dissertation: A quantitative examination of research methods and software in scientific publications
\item Advisor: Dr. Erjia Yan
\end{itemize} 

{\sl Master of Science in Library and Information Science} \hfill 2012 - 2013 \\
School of Information Studies\\
\textbf{Syracuse University}, Syracuse, NY
 
{\sl Bachelor’s Degree in History} \hfill 2003 - 2007\\
School of History\\
\textbf{Beijing Normal University}, Beijing, China

% Publications section
\section{PUBLICATIONS}

\textbf{Peer-Reviewed Journal Articles}

[J21] Jiao, C., \underline{Li, K.}, \& Fang, Z. (2022). Data sharing practices across knowledge domains: a dynamic examination of data availability statements in PLOS ONE publications \textit{Journal of Information Science}, OnlineFirst. \href{https://doi.org/10.1177/0165551522110183}{\nolinkurl{https://doi.org/10.1177/0165551522110183}} 

[J20] \underline{Li, K.} (2022). Relationships between Method-section citation rates and citation contexts: Evidence from highly cited references in psychology. \textit{Online Information Review}, 46(5): 829-845 \href{https://doi.org/10.1108/OIR-03-2021-0171}{\nolinkurl{https://doi.org/10.1108/OIR-03-2021-0171}} 

[J19] Wang Y., Zhang C., \underline{Li K.} (2022) A review on method entities in the academic literature: extraction, evaluation, and application. \textit{Scientometrics}. 127: 2479-2520. \href{https://doi.org/10.1007/s11192-022-04332-7}{\nolinkurl{https://doi.org/10.1007/s11192-022-04332-7}}

[J18] Ma, R., \& \underline{Li, K.} (2022). Digital humanities as a cross-disciplinary battleground: An examination of inscriptions in journal publications. \textit{Journal of the Association for Information Science and Technology}, 73(2): 172-187. \href{https://doi.org/10.1002/asi.24534}{\nolinkurl{https://doi.org/10.1002/asi.24534}}

[J17] \underline{Li, K.} \& Jiao, C. (2022). The data paper as a socio-linguistic epistemic object: A content analysis on the rhetorical moves used in data paper abstracts. \textit{Journal of the Association for Information Science and Technology.}, 73(6): 834-846. \href{https://doi.org/10.1002/asi.24585}{\nolinkurl{https://doi.org/10.1002/asi.24585}}

[J16] \underline{Li, K.}, Jiao, C., Sugimoto, C. R., \& Lariviere, V. (2022). Versioning boundary objects: the citation profile of the Diagnostic and Statistical Manual for Mental Disorders (DSM). \textit{Journal of Documentation}, 78(4): 871-889. \href{https://doi.org/10.1108/JD-06-2021-0117}{\nolinkurl{https://doi.org/10.1108/JD-06-2021-0117}}

[J15] Lou, W. Su, Z., He, J. \& \underline{Li, K.} (2021) A temporally dynamic examination of research method usage in the Chinese library and information science community. \textit{Information Processing \& Management}, 58(5): 102686. \href{https://doi.org/10.1016/j.ipm.2021.102686}{\nolinkurl{https://doi.org/10.1016/j.ipm.2021.102686}} 

[J14] Wu C., Yan, E., Zhu, Y., \& \underline{Li, K.} (2021). Gender imbalance in the productivity of funded projects: A study of the outputs of National Institutes of Health R01 grants. \textit{Journal of the Association for Information Science and Technology}, 72(11): 1386-1399. \href{https://doi.org/10.1002/asi.24487}{\nolinkurl{https://doi.org/10.1002/asi.24487}}

[J13] \underline{Li, K.} (2021) The re-instrumentalization of the Diagnostic and Statistical Manual of Mental Disorders (DSM) in psychological publications: a citation context analysis. \textit{Quantitative Science Studies}, 2(2): 678-697. \href{https://doi.org/10.1162/qss_a_00124}{\nolinkurl{https://doi.org/10.1162/qss_a_00124}}

[J12] Lou, W., Zhang, J., \underline{Li, K.}, \& He, J. (2020). Understanding the Application of Science Mapping Tools in LIS and Non-LIS Domains. \textit{Data and Information Management}, 4(2), 94-108. \href{https://doi.org/10.2478/dim-2020-0006}{\nolinkurl{https://doi.org/10.2478/dim-2020-0006}}

[J11] Yan, E., Chen, Z. \& \underline{Li, K.} (2020). The relationship between journal citation impact and citation sentiment: A study of 32 million citances in PubMed Central, \textit{Quantitative Science Studies}, 1(2): 664-674. \href{https://doi.org/10.1162/qss\_a\_00040}{\nolinkurl{https://doi.org/10.1162/qss_a_00040}}

[J10] Yan, E., Chen, Z. \& \underline{Li, K.} (2020). Authors' status and the perceived quality of their work: Measuring citation sentiment change in nobel articles. \textit{Journal of the Association for Information Science and Technology}, 71(3): 314-324. \href{https://doi.org/10.1002/asi.24237}{\nolinkurl{https://doi.org/10.1002/asi.24237}}

[J9] \underline{Li, K.}, Greenberg, J. and Dunic, J. (2020). Data objects and documenting scientific processes: An analysis of data events in biodiversity data papers. \textit{Journal of the Association for Information Science and Technology}, 71(2): 172-182. \href{https://doi.org/10.1002/asi.24226}{\nolinkurl{https://doi.org/10.1002/asi.24226}}

[J8] \underline{Li, K.}, Chen, P.-Y., \& Yan, E. (2019). Challenges of measuring software impact through citations: An examination of the lme4 R package. \textit{Journal of Informetrics}, 13(1), 449-461. \href{https://doi.org/10.1016/j.joi.2019.02.007}{\nolinkurl{https://doi.org/10.1016/j.joi.2019.02.007}}

[J7] \underline{Li, K.}, \& Yan, E. (2019). Are NIH-funded publications fulfilling the proposed research? An examination of concept-matchedness between NIH research grants and their supported publications. \textit{Journal of Informetrics}, 13(1), 226-237. \href{https//doi.org/10.1016/j.joi.2019.01.001}{\nolinkurl{https//doi.org/10.1016/j.joi.2019.01.001}}

[J6] Li, Y., Wu, C., Yan, E., \& \underline{Li, K.} (2018). Will open access increase journal CiteScores? An empirical investigation over multiple disciplines. \textit{PLOS ONE}, 13(8): e0201885. \href{https://doi.org/10.1371/journal.pone.0201885}{\nolinkurl{https://doi.org/10.1371/journal.pone.0201885}}

[J5] Yan, E., \& \underline{Li, K.} (2018). Which domains do open-access journals do best in? A 5-year longitudinal study. \textit{Journal of the Association for Information Science and Technology}, 69(6): 844-856. \href{https://doi.org/10.1002/asi.24002}{\nolinkurl{https://doi.org/10.1002/asi.24002}}

[J4] \underline{Li, K.}, Rollins, J., \& Yan, E. (2018). Web of Science use in published research and review papers 1997–2017: a selective, dynamic, cross-domain, content-based analysis. \textit{Scientometrics}, 115(1), 1-20. \href{https://doi.org/10.1007/s11192-017-2622-5}{\nolinkurl{https://doi.org/10.1007/s11192-017-2622-5}} 

[J3] Zhao, M., Yan, E., \& \underline{Li, K.} (2018). Data set mentions and citations: A content analysis of full- text publications. \textit{Journal of the Association for Information Science and Technology}, 69(1): 32-46. \href{https://doi.org/10.1002/asi.23919}{\nolinkurl{https://doi.org/10.1002/asi.23919}}

[J2] \underline{Li, K.} \& Yan, E. (2018). Co-mention network of R packages: Scientific impact and clustering structure. \textit{Journal of Informetrics}, 12(1): 87-100. \href{https://doi.org/10.1016/j.joi.2017.12.001}{\nolinkurl{https://doi.org/10.1016/j.joi.2017.12.001}}

[J1] \underline{Li, K.}, Yan, E., \& Feng, Y. (2017). How is R cited in research outputs? Structure, impacts, and citation standard. \textit{Journal of Informetrics}, 11(4): 989-1002. \href{https://doi.org/10.1016/j.joi.2017.08.003}{\nolinkurl{https://doi.org/10.1016/j.joi.2017.08.003}}

\textbf{Refereed Full Papers in Conference Proceedings}

[C4] \underline{Li, K.}, \& Yan, E. (2019). Using a keyword extraction pipeline to understand concepts in future work sections of research papers. In \textit{17th International Conference on Scientometrics and Informetrics}.

[C3] Feng, Y., \underline{Li, K.}, \& Agosto, D.E. (2017). Healthy users’ personal health information management from activity trackers: The perspective of gym-goers. In \textit{Proceedings of the Annual Meeting of Association for Information Science and Technology} (54: pp. 71-81). \href{https://doi.org/10.1002/pra2.2017.14505401009}{\nolinkurl{https://doi.org/10.1002/pra2.2017.14505401009}}

[C2] \underline{Li, K.}, Greenberg, J., \& Lin, X. (2016). Software Citation, Reuse and Metadata Considerations: An Exploratory Study Examining LAMMPS. In \textit{Proceedings of the Annual Meeting of the Association for Information Science and Technology} (53: pp. 1-10). \href{https://doi.org/10.1002/pra2.2016.14505301072}{\nolinkurl{https://doi.org/10.1002/pra2.2016.14505301072}}

[C1] Qin, J., \& \underline{Li, K.} (2013). How portable are the metadata standards for scientific data? a proposal for a metadata infrastructure. In \textit{Proceedings of the International Conference on Dublin Core and Metadata Applications} (pp. 25–34). \href{https://dcpapers.dublincore.org/pubs/article/view/3670}{\nolinkurl{https://dcpapers.dublincore.org/pubs/article/view/3670}}

\textbf{Refereed Short Papers in Conference Proceedings}

[S6] \underline{Li, K.}, Chen, J., \& Ni, C. (2023). The patterns of publication languages used by STEM research in China. In \textit{19th International Conference on Scientometrics and Informetrics}.

[S5] \underline{Li, K.} \& Jiao, C. (2021). How are data paper abstracts constructed? Preliminary analysis of rhetorical moves in data paper abstracts from Scientific Data and Data in Brief. In \textit{18th International Conference on Scientometrics and Informetrics}.

[S4] Dagiene, E. \& \underline{Li, K.} (2021). ISBNs as identifiers for books in research evaluations. In \textit{18th International Conference on Scientometrics and Informetrics}.

[S3] Ma, R., \underline{Li, K.}, \& He, D. (2021). Understanding the Narrative Functions of Visualization in Digital Humanities Publications: A Case Study of the Journal of Cultural Analytics. In \textit{Proceedings of iConference 2021}.

[S2] Pascua, S., \underline{Li, K.}, \& Greenberg, J. (2019). Toward A Metadata Activity Matrix: Conceptualizing and Grounding the Research Life-cycle and Metadata Connections. In \textit{Proceedings of the International Conference on Dublin Core and Metadata Applications}.

[S1] Guerra, J., Wei, Q., \underline{Li, K.}, Ahumada, L., Winston, F., \& Desai, B. R. (2018). Scosy: A Biomedical Collaboration Recommendation System. Presented at the 40th Annual International Conference of the IEEE Engineering in Medicine and Biology Society (EMBC), Honolulu, HI, USA.

\textbf{Refereed Conference Posters}

[P8] Chen, PY, \underline{Li, K.} \& Jiao, C. (2022), A Preliminary Analysis of Geography of Collaboration in Data Papers by S\&T Capacity Index. \textit{Proceedings of the Association for Information Science and Technology}, 59: 642-644.

[P7] Ma, R.Q. \& \underline{Li, K.} (2020). Telling Multifaceted Stories with Humanities Data: Visualizing Book of Hours Manuscripts. Presented at iConference 2020 Annual Meeting. (Finalist for Best Poster Award).

[P6] \underline{Li, K.}, Chen, P.-Y., \& Fang, Z. (2019). Disciplinarity of Software Papers: A Preliminary Analysis. In \textit{Proceedings of the Annual Meeting of Association for Information Science and Technology}.

[P5] \underline{Li, K.}, \& Chen, P.-Y. (2019). A preliminary scientometric analysis of the Cross-Strait scientific collaboration. In \textit{17th International Conference on Scientometrics and Informetrics}.

[P4] He, J. \& \underline{Li, K.} (2019). How comprehensive is the PubMed Central Open Access full-text database?. Presented at iConference 2019 Annual Meeting, Washington, DC, USA. \href{http://hdl.handle.net/2142/103317}{http://hdl.handle.net/2142/103317}

[P3] \underline{Li, K.} \& Chen, PY. (2018). The narrative structure as a citation context in data papers: A preliminary analysis of Scientific Data. In \textit{Proceedings of the Annual Meeting of Association for Information Science and Technology}. \href{https://doi.org/10.1002/pra2.2018.14505501147}{\nolinkurl{https://doi.org/ 10.1002/pra2.2018.14505501147}}

[P2] \underline{Li, K.} \& Xu, S. (2017). Measuring the impact of R packages. In \textit{Proceedings of the Annual Meeting of Association for Information Science and Technology} (54: pp. 739-741). \href{https://doi.org/10.1002/pra2.2017.14505401138}{\nolinkurl{https://doi.org/10.1002/pra2.2017.14505401138}}

[P1] \underline{Li, K.} (2016). How much can data citation standards be used for scientific software? A crosswalk analysis of data citation standards for software citation needs. Presented in Research Data Alliance 8th Plenary Meeting, Denver, Colorado, September 14-17, 2016.

\textbf{Book Reviews}

[B1] \underline{Li, K.} (2019). A review of The Politics of Mass Digitization. \textit{Education for Information}, 35, 361-363. \href{https://doi.org/10.3233/EFI-199011}{\nolinkurl{https://doi.org/10.3233/EFI-199011}}

\textbf{Doctoral Colloquia (Lightly Peer-Reviewed)}

[D2] \underline{Li, K.} (2019). Quantitative examination of research methods and software in scientific publications. Selected to attend the Doctoral Colloquium at ASIS\&T 2019, Melbourne, Australia, October 19, 2019.

[D1] \underline{Li, K.} (2019). What is the discplinarity of software. Selected to attend the Doctoral Forum at 17th International Conference on Scientometrics and Informetrics, Rome, Italy, September 2, 2019.

\textbf{Non-Refereed Conference Presentations}

[T1] \underline{Li, K.} (2012). RDA in China. Presented at ALA Annual Conference 2012. June 24, 2012. Anaheim, California.

\section{TEACHING EXPERIENCE}

\textbf{Instructor}

School of Information Resource Management, Renmin University of China
\begin{itemize}
\item \textbf{Information Visualization in the Context of Digital Humanities} (master level; in Chinese-language) (Fall '21, '22)
\item \textbf{Information Visualization} (undergraduate level; in Chinese-language) (Spring '22)
\item \textbf{Knowledge Organization} (master level; in Chinese-language; in Chinese-language) (Spring '22)
\end{itemize}

College of Computing \& Informatics, Drexel University

\begin{itemize}
\item \textbf{INFO 250: Information Visualization} (Online) (Summer '20)
\item \textbf{INFO 590: Foundations of Data and Information} (Online) (Winter '20)
\item \textbf{INFO 250: Information Visualization} (On-site) (Summer '19)
\end{itemize}

\section{HONORS \& AWARDS}

{\sl LIS Education and Data Science for the National Digital Platform (LEADS-4-NDP) Fellowship}  \hfill 2019 \\
\textbf{Drexel University}
\begin{itemize}
\item Project: Automatic Identification of Publisher Entities to Support Discovery and Navigation (with OCLC)
\item Awarded \$5,000 plus travel support
\item Funded by the Institute of Museum and Library Services
\end{itemize} 

{\sl RDA/US Data Share Fellowship}  \hfill 2016 - 2017 \\
\textbf{Research Data Alliance}
\begin{itemize}
\item Project: Metadata Standards and Software Citation Practices
\item Awarded \$10,000 plus travel support
\item Funded by the Alfred P. Sloan Foundation
\end{itemize} 

{\sl Student Travel Award} \hfill 2019 \\
\textbf{The International Society for Informetrics and Scientometrics}

% Professional Experience
\section{PROFESSIONAL EXPERIENCE}

\textit{Cataloger} \hfill 2014 - 2015\\
\textbf{Ingram Content Group, Fort Wayne, Indiana}

\textit{Cataloging Librarian} \hfill 2007 - 2012\\
\textbf{Capital Library of China, Beijing, China}

% Service and Membership
\section{SERVICE \& MEMBERSHIPS}

\textbf{Service}
\begin{itemize}
\item \textbf{Editorial Board Member}, Annual Review of Information Science and Technology (ARIST), 2022-2024
\item \textbf{President-Elect}, ASIS\&T SIG-MET, 2022-2023
\item \textbf{Jury}, ASIS\&T Lois Lunin Award, 2022, 2023
\item \textbf{Mentor}, ASIS\&T Mentor Program, 2022-2023
\item \textbf{Webmaster}, ASIS\&T SIG-MET, 2021-2022
\item \textbf{Committee Member}, ISSI 2021 Paper of the Year Award Committee, 2021
\item \textbf{Secretary}, ASIS\&T SIG-MET, 2020-2021
\item \textbf{Award Coordinator}, ASIS\&T SIG-MET, 2019-2020
\end{itemize}

\textbf{Selected Peer Review Experience}:  
(My full peer review record can be retrieved from my ORCID page.)

\begin{itemize}
\item Journal of the Association for Information Science and Technology
\item Quantitative Science Studies
\item Scientometrics
\item Journal of Informetrics
\item Journal of Information Science
\item PLoS ONE
\end{itemize}

\textbf{Membership}
\begin{itemize}
\item The Association for Information Science and Technology (ASIS\&T)
\end{itemize}


%----------------------------------------------------------------------------------------

\end{resume}
\end{document}