%%%%%%%%%%%%%%%%%%%%%%%%%%%%%%%%%%%%%%%%%
% Medium Length Graduate Curriculum Vitae
% LaTeX Template
% Version 1.1 (9/12/12)
%
% This template has been downloaded from:
% http://www.LaTeXTemplates.com
%
% Original author:
% Rensselaer Polytechnic Institute (http://www.rpi.edu/dept/arc/training/latex/resumes/)
%
% Important note:
% This template requires the res.cls file to be in the same directory as the
% .tex file. The res.cls file provides the resume style used for structuring the
% document.
%
%%%%%%%%%%%%%%%%%%%%%%%%%%%%%%%%%%%%%%%%%

%----------------------------------------------------------------------------------------
%	PACKAGES AND OTHER DOCUMENT CONFIGURATIONS
%----------------------------------------------------------------------------------------

\documentclass[margin, 10pt]{res} % Use the res.cls style, the font size can be changed to 11pt or 12pt here
\usepackage{hyperref}

\setlength{\textwidth}{5in} % Text width of the document
\pagestyle{plain}

\begin{document}

%----------------------------------------------------------------------------------------
%	NAME AND ADDRESS SECTION
%----------------------------------------------------------------------------------------

\moveleft.5\hoffset\centerline{\large\bf Kai Li} % Your name at the top
 
\moveleft\hoffset\vbox{\hrule width\resumewidth height 1pt}\smallskip % Horizontal line after name; adjust line thickness by changing the '1pt'
 
\moveleft.5\hoffset\centerline{Office 1195, 3675 Market St., Philadelphia, PA 19104} % Your address
\moveleft.5\hoffset\centerline{\href{mailto:kl696@drexel.edu}{kl696@drexel.edu}}
\moveleft.5\hoffset\centerline{\href{http://kaili.us}{http://kaili.us}}

%----------------------------------------------------------------------------------------

\begin{resume}

%----------------------------------------------------------------------------------------
%	OBJECTIVE SECTION
%----------------------------------------------------------------------------------------
 
\section{RESEARCH INTERESTS}  

Library and information science; Data science; Scholarly communication; Scientometrics; Open science; Data and software citation; Information organization; Critical data studies; Science and technology studies; Data mining; Text mining; Machine learning; Natural language processing

%----------------------------------------------------------------------------------------
%	Technology SKILLS SECTION
%----------------------------------------------------------------------------------------
 
\section{EDUCATION}

{\sl Ph.D. in Information Studies} \hfill 2015 - Present \\
College of Computing \& Informatics\\
\textbf{Drexel University}, Philadelphia, PA
\begin{itemize}
\item Proposal defended in March, 2019. Expect to graduate in Spring, 2020.
\item Dissertation: A quantitative examination of research methods and software in scientific publications
\item Committee: Dr. Erjia Yan (Chair), Dr. Jane Greenberg, Dr. Ali Kenner, Dr. Cassidy Sugimoto, Dr. Chaomei Chen, and Dr. Jake Williams
\end{itemize} 

{\sl Master of Science in Library and Information Science} \hfill 2012 - 2013 \\
School of Information Studies\\
\textbf{Syracuse University}, Syracuse, NY
 
{\sl Bachelor’s Degree in History} \hfill 2003 - 2007\\
School of History\\
\textbf{Beijing Normal University}, Beijing, China

% Awards section

\section{HONORS \& AWARDS}

{\sl LIS Education and Data Science for the National Digital Platform (LEADS-4-NDP) Fellowship}  \hfill 2019 \\
\textbf{Drexel University}
\begin{itemize}
\item Project: Automatic Identification of Publisher Entities to Support Discovery and Navigation (with OCLC)
\item Awarded \$5,000 plus travel support
\end{itemize} 

{\sl RDA/US Data Share Fellowship}  \hfill 2016 - 2017 \\
\textbf{Research Data Alliance}
\begin{itemize}
\item Project: Metadata Standards and Software Citation Practices
\item Awarded \$10,000 plus travel support
\end{itemize} 

{\sl Student Travel Award} \hfill 2019 \\
\textbf{The International Society for Informetrics and Scientometrics}

{\sl Travel Awards} \hfill 2016, 2017 \\
\textbf{Department of Information Science, Drexel University}

\section{RESEARCH EXPERIENCE}

\textbf{College of Computing \& Informatics}, Drexel University\\
\begin{itemize}
     \item Research Assistant (with Dr. Erjia Yan) \hfill 2017 - 2019
     \begin{itemize}
     	\item  Project: “Building an Entity-based Research Framework to Enhance Digital Services on Knowledge Discovery and Delivery” (funded by the Institute of Museum and Library Services)
   	\end{itemize}
   	\item Research Assistant (with Dr. Jane Greenberg) \hfill 2016 - 2017
     \begin{itemize}
     	\item  Project: “Transforming Data Adaptation Science and Service: An Innovative Visual Ontology Application” (funded by Center for Visual \& Decision Informatics)
   	\end{itemize}
\end{itemize}

\textbf{College of Information Studies}, Syracuse University\\
\begin{itemize}
	\item Research Assistant (with Dr. Kevin Crowston \& Dr. Carsten Oesterlund) \hfill 2013
    \begin{itemize}
     	\item Project: “Collaborative Research: Focusing Attention to Improve the Performance of Citizen Science Systems - Beautiful Images and Perspective Observers” (funded by the National Science Foundation)
    \end{itemize}
    \item Research Assistant (with Dr. Jian Qin) \hfill 2012 - 2013
\end{itemize}

% Publications section
\section{PUBLICATIONS}

\textbf{Peer-Reviewed Journal Articles}

[J10] Yan, E., Chen, Z., \& \underline{Li, K.} (2019). Authors’ status and the perceived quality of their work: measuring citation sentiment change in Nobel papers. \textit{Journal of the Association for Information Science and Technology}. Advance online publication. \href{https://doi.org/10.1002/asi.24237}{https://doi.org/10.1002/asi.24237}

[J9] \underline{Li, K.}, Greenberg, J., \& Dunic, J. (2019). Data objects and documenting scientific processes: An analysis of data events in biodiversity data papers. \textit{Journal of the Association for Information Science and Technology}. Advance online publication. \href{https://doi.org/10.1002/asi.24226}{https://doi.org/10.1002/asi.24226}

[J8] \underline{Li, K.}, Chen, P.-Y., \& Yan, E. (2019). Challenges of measuring software impact through citations: An examination of the lme4 R package. \textit{Journal of Informetrics}, 13(1), 449–461. \href{https://doi.org/10.1016/j.joi.2019.02.007}{https://doi.org/10.1016/j.joi.2019.02.007}

[J7] \underline{Li, K.}, \& Yan, E. (2019). Are NIH-funded publications fulfilling the proposed research? An examination of concept-matchedness between NIH research grants and their supported publications. \textit{Journal of Informetrics}, 13(1), 226–237. \href{https//doi.org/10.1016/j.joi.2019.01.001}{https//doi.org/10.1016/j.joi.2019.01.001}

[J6] Li, Y., Wu, C., Yan, E., \& \underline{Li, K.} (2018). Will open access increase journal CiteScores? An empirical investigation over multiple disciplines. \textit{PLOS ONE}, 13(8), e0201885. \href{https://doi.org/10.1371/journal.pone.0201885}{https://doi.org/10.1371/journal.pone.0201885}

[J5] Yan, E., \& \underline{Li, K.} (2018). Which domains do open-access journals do best in? A 5-year longitudinal study. \textit{Journal of the Association for Information Science and Technology}, 69(6), 844–856. \href{https://doi.org/10.1002/asi.24002}{https://doi.org/10.1002/asi.24002}

[J4] \underline{Li, K.}, Rollins, J., \& Yan, E. (2018). Web of Science use in published research and review papers 1997–2017: a selective, dynamic, cross-domain, content-based analysis. \textit{Scientometrics}, 115(1), 1–20. \href{https://doi.org/10.1007/s11192-017-2622-5}{https://doi.org/10.1007/s11192-017-2622-5}

[J3] Zhao, M., Yan, E., \& \underline{Li, K.} (2018). Data set mentions and citations: A content analysis of full- text publications. \textit{Journal of the Association for Information Science and Technology}, 69(1), 32–46. \href{https://doi.org/10.1002/asi.23919}{https://doi.org/10.1002/asi.23919}

[J2] \underline{Li, K.} \& Yan, E. (2018). Co-mention network of R packages: Scientific impact and clustering structure. \textit{Journal of Informetrics}, 12(1), 87-100. \href{https://doi.org/10.1016/j.joi.2017.12.001}{https://doi.org/ 10.1016/j.joi.2017.12.001}

[J1] \underline{Li, K.}, Yan, E., \& Feng, Y. (2017). How is R cited in research outputs? Structure, impacts, and citation standard. \textit{Journal of Informetrics}, 11(4), 989–1002. \href{https://doi.org/10.1016/j.joi.2017.08.003}{https://doi.org/10.1016/j.joi.2017.08.003}

\textbf{Refereed Full Papers in Conference Proceedings}

[C4] \underline{Li, K.}, \& Yan, E. (2019). Using a keyword extraction pipeline to understand concepts in future work sections of research papers. In \textit{17th International Conference on Scientometrics and Informetrics}.

[C3] Feng, Y., \underline{Li, K.}, \& Agosto, D.E. (2017). Healthy users’ personal health information management from activity trackers: The perspective of gym-goers. In \textit{Proceedings of the Annual Meeting of Association for Information Science and Technology} (54: pp. 71-81). \href{https://doi.org/10.1002/pra2.2017.14505401009}{https://doi.org/10.1002/pra2.2017.14505401009}

[C2] \underline{Li, K.}, Greenberg, J., \& Lin, X. (2016). Software Citation, Reuse and Metadata Considerations: An Exploratory Study Examining LAMMPS. In \textit{Proceedings of the Annual Meeting of the Association for Information Science and Technology} (53: pp. 1-10). \href{https://doi.org/10.1002/pra2.2016.14505301072}{https://doi.org/10.1002/pra2.2016.14505301072}

[C1] Qin, J., \& \underline{Li, K.} (2013). How portable are the metadata standards for scientific data? a proposal for a metadata infrastructure. In \textit{Proceedings of the International Conference on Dublin Core and Metadata Applications} (pp. 25–34). \href{https://dcpapers.dublincore.org/pubs/article/view/3670}{https://dcpapers.dublincore.org/pubs/article/view/3670}

\textbf{Refereed Short Papers in Conference Proceedings}

[S2] Pascua, S., \underline{Li, K.}, \& Greenberg, J. (2019). Toward A Metadata Activity Matrix: Conceptualizing and Grounding the Research Life-cycle and Metadata Connections. In \textit{Proceedings of the International Conference on Dublin Core and Metadata Applications}.

[S1] Guerra, J., Wei, Q., \underline{Li, K.}, Ahumada, L., Winston, F., \& Desai, B. R. (2018). Scosy: A Biomedical Collaboration Recommendation System. Presented at the 40th Annual International Conference of the IEEE Engineering in Medicine and Biology Society (EMBC), Honolulu, HI, USA.

\textbf{Refereed Conference Posters}

[P6] \underline{Li, K.}, Chen, P.-Y., \& Fang, Z. (2019). Disciplinarity of Software Papers: A Preliminary Analysis. In \textit{Proceedings of the Annual Meeting of Association for Information Science and Technology}.

[P5] \underline{Li, K.}, \& Chen, P.-Y. (2019). A preliminary scientometric analysis of the Cross-Strait scientific collaboration. In \textit{17th International Conference on Scientometrics and Informetrics}.

[P4] He, J. \& \underline{Li, K.} (2019). How comprehensive is the PubMed Central Open Access full-text database?. Presented at iConference 2019 Annual Meeting, Washington, DC, USA. \href{http://hdl.handle.net/2142/103317}{http://hdl.handle.net/2142/103317}

[P3] \underline{Li, K.} \& Chen, PY. (2018). The narrative structure as a citation context in data papers: A preliminary analysis of Scientific Data. In \textit{Proceedings of the Annual Meeting of Association for Information Science and Technology}. \href{https://doi.org/10.1002/pra2.2018.14505501147}{https://doi.org/ 10.1002/pra2.2018.14505501147}

[P2] \underline{Li, K.} \& Xu, S. (2017). Measuring the impact of R packages. In \textit{Proceedings of the Annual Meeting of Association for Information Science and Technology} (54: pp. 739-741). \href{https://doi.org/10.1002/pra2.2017.14505401138}{https://doi.org/10.1002/pra2.2017.14505401138}

[P1] \underline{Li, K.} (2016). How much can data citation standards be used for scientific software? A crosswalk analysis of data citation standards for software citation needs. Presented in Research Data Alliance 8th Plenary Meeting, Denver, Colorado, September 14-17, 2016.

\textbf{Book Reviews}

[B1] \underline{Li, K.} (2019). A review of The Politics of Mass Digitization. \textit{Education for Information}, 35, 361-363. \href{https://doi.org/10.3233/EFI-199011}{https://doi.org/10.3233/EFI-199011}

\textbf{Doctoral Colloquia (Lightly Peer-Reviewed)}

[D2] \underline{Li, K.} (2019). Quantitative examination of research methods and software in scientific publications. Selected to attend the Doctoral Colloquium at ASIS\&T 2019, Melbourne, Australia, October 19, 2019.

[D1] \underline{Li, K.} (2019). What is the displinarity of software. Selected to attend the Doctoral Forum at 17th International Conference on Scientometrics and Informetrics, Rome, Italy, September 2, 2019.

\textbf{Non-Refereed Conference Presentations}

[T1] \underline{Li, K.} (2012). RDA in China. Presented at ALA Annual Conference 2012. June 24, 2012. Anaheim, California.

\section{TEACHING EXPERIENCE}

\textbf{Instructor}

College of Computing \& Informatics, Drexel University

\begin{itemize}
\item \textbf{INFO 250 (On-site): Information Visualization} (Summer '19)
\end{itemize}

\textbf{Teaching Assistant}

College of Computing \& Informatics, Drexel University

\begin{itemize}
\item \textbf{INFO 659: Introduction to Data Analytics} (Fall '19)
\item \textbf{INFO 250: Information Visualization} (Spring '19)
\item \textbf{INFO 683: Resources for Children} (Summer '18)
\item \textbf{CI 101: Introduction to Computing and Informatics} (Fall '15)
\end{itemize}

% Professional Experience
\section{PROFESSIONAL EXPERIENCE}

\textit{Research Intern} \hfill 2016 - 2017\\
\textbf{The Children’s Hospital of Philadelphia, Philadelphia, PA}

\textit{Cataloger} \hfill 2014 - 2015\\
\textbf{Ingram Content Group, Fort Wayne, Indiana}

\textit{Cataloging Librarian} \hfill 2007 - 2012\\
\textbf{Capital Library of China, Beijing, China}

% Service and Membership
\section{SERVICE \& MEMBERSHIPS}

\textbf{Reviewer}
\begin{itemize}
\item Journal of the Association for Information Science and Technology
\item Journal of Informetrics
\item Scientometrics
\item Library Hi Tech
\item IEEE Access
\item iConference Annual Meeting ('19)
\item The American Medical Informatics Association (AMIA) Annual Symposium ('18 '19)
\end{itemize}

\textbf{Membership}
\begin{itemize}
\item The Association for Information Science and Technology (ASIS\&T)
\end{itemize}

% Technological Skills
\section{TECHNICAL SKILLS}
\begin{itemize}
\item \textbf{Programming}: Python, R
\item \textbf{Database}: MySQL
\item \textbf{Markup Language}: HTML, CSS, XML, XSLT, LATEX
\end{itemize}


%----------------------------------------------------------------------------------------

\end{resume}
\end{document}